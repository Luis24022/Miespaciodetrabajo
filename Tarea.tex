\documentclass[a4paper,10pt]{article}
\usepackage[utf8]{inputenc}

%opening
\title{Mi primer documento de latex} % Cambia el titulo por alguno que tu elijas.
\author{Luis Angel Romero Gonzalez} % Cambialo por tu nombre completo.

\begin{document}

\maketitle

% \begin{abstract}
% 
% \end{abstract}

\section{Seccion 1}

Bienvenido, este es mi primer documento en un editor de texto en la clase de computación \\ % Cambia este texto por algo tuyo.

No tengo mucha idea de que escribir aqui, aun no se manejar este programa, pero me esforzare por aprenderlo y emplearlo para mis trabajos\\ % Cambia este texto por algun texto tuyo.

$ e^{i \pi} + 1 = 0 $  Esta es la ecuacion conocida como la identidad de Euler la cual relaciona la funcion exponencial con las funciones trigonometricas y para ello uso a i. Si consideramos el caso en el que x es igual a pi, el coseno de pi seria -1, el senop de pi es cero lo cual nos relacionaria que la exponencial e elevada a la potencia i multiplicada por pi y a eso sumarle uno es igual a cero% Comenta esta ecacución y escribe abajo una ecuación muy simple (la que quieras).

$ x = \dfrac{-b \pm \sqrt{b^2 - 4ac}}{2a}$ Esta es la formula general o chicharronera que se emplea para buscar la solucion a ecuaciones de segundo grado


\end{document}
